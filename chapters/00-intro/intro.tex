\chapter*{Вступ}

\paragraph{Актуальність теми}

Сьогодні перед дослідниками та науковцями все частіше постають задачі аналізу даних та здійснення прогнозів чи прийняття рішень на підставі результатів аналізу. Із розвитком обчислювальної техніки збільшуються також об'єми баз даних, що піддаються аналізу. Виникає необхідність створення оптимальних алгоритмів обробки великих масивів даних.
Одним із важливих та найбільш ресурсоємних етапів такої обробки є кластеризація.

\paragraph{Мета і завдання дослідження}

Метою дослідження є розвиток методики кластеризації великих масивів даних.
Для досягнення цієї мети були сформульовані та вирішені такі основні завдання:
\begin{itemize}
    \item провести детальний аналіз та дослідити ефективність різноманітних алгоритмів кластеризації;
    \item здійснити контроль якості результатів роботи алгоритмів;
    \item розробити методи та виявити можливі оптимізації, що дозволять прискорити виконання задачі кластеризації
\end{itemize}

\subparagraph{Об'єкт дослідження}
Об'єктом дослідження є методи ієрархічні та плоскі алгоритми кластеризації даних.
