\chapter*{Вступ}

\paragraph{Актуальність теми}

Сьогодні все частіше виникають задачі, так чи інакше пов'язані із розбиттям масиву об'єктів на групи за певними критеріями. Із розвитком обчислювальної техніки збільшуються також об'єми баз даних, що можуть піддаватись такому аналізу. Виникає необхідність створення оптимальних алгоритмів обробки великих масивів даних.

\paragraph{Мета і завдання дослідження}

Метою дослідження є розвиток методики кластеризації великих масивів даних.
Для досягнення цієї мети були сформульовані та вирішені такі основні завдання:
\begin{itemize}
    \item провести детальний аналіз та дослідити ефективність різноманітних алгоритмів кластеризації;
    \item здійснити контроль якості результатів роботи алгоритмів;
    \item розробити методи та виявити можливі оптимізації, що дозволять прискорити виконання задачі кластеризації
\end{itemize}

\subparagraph{Об'єкт дослідження}
Об'єктом дослідження є методи ієрархічні та плоскі алгоритми кластеризації даних.

\subparagraph{Предмет дослідження}
Предметом дослідження є придатність алгоритмів кластеризації до використання на великих об'ємах даних.
