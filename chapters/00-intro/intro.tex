\chapter*{Вступ}

\paragraph{Актуальність теми}

Сьогодні все частіше виникають задачі, так чи інакше пов'язані із розбиттям масиву об'єктів на групи за певними критеріями. Із розвитком обчислювальної техніки збільшуються також об'єми баз даних, що можуть піддаватись такому аналізу. Виникає необхідність створення оптимальних алгоритмів опрацювання великих масивів даних.

\paragraph{Мета і завдання дослідження}

Метою дослідження є оптимізація алгоритмів кластеризації для обробки великих масивів даних.
Для досягнення цієї мети були сформульовані та вирішені такі основні завдання:
\begin{itemize}
    \item провести порівняльний аналіз алгоритмів кластеризації
    \item розробити методи та виявити покращення, що дозволять прискорити виконання задачі кластеризації
    \item здійснити оцінку покращення роботи алгоритмів кластеризації за рахунок здійсненої оптимізації
\end{itemize}

\subparagraph{Об'єкт дослідження}
Об’єктом дослідження є процеси ієрархічної та роздільної кластеризації даних великих об’ємів.

\subparagraph{Предмет дослідження}
Предметом дослідження є оптимізація алгоритмів кластеризації для ефективного аналізу даних великих об’ємів.
