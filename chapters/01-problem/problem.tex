\chapter{Огляд стану проблеми та основні поняття}\label{ch:01}

Термін "кластеризація" вперше використав Тріон у \cite{Tryon:Cluster:1939}

Кластеризацією називається процес розбиття певної заданої множини об'єктів на підмножини, що називаються \textit{кластерами}, так, щоб в одній підмножині (кластері) опинились об'єкти, якимось чином схожі між собою. Ці об'єкти також називають \textit{спостереженнями}.
Кожне спостереження характеризується певним скінченним набором атрибутів. Якщо у кожного об'єкта є точно n атрибутів, і якщо всі ці атрибути у якийсь спосіб приведено до числового вигляду, то можна розцінювати такий масив даних як набір точок у n-вимірному просторі.
\subsection{Задачі кластеризації}
За останні кілька десятиліть виникла ціла низка задач, розв'язання яких на певному етапі вимагає кластеризації вхідних даних. Ці задачі походять із різноманітних сфер людської діяльності -- соціології, медицини, математики, і навіть із комерційної діяльності. Кластеризація зазвичай застосовується при розпізнаванні образів, виявленні прихованих закономірностей даних, та ін. 
\\
Для просторів об'єктів, розмірність яких не перевищує 2, людина здатна провести якісну кластеризацію вручну. Для цього достатньо представити вхідні дані як масив точок на двовимірній площині або одновимірній прямій (в залежності від розмірності простору). Таким чином зображені дані піддаються швидкому візуальному аналізу, що дозволяє ефективно встановити основні кластери (або ж неможливість кластеризації даного набору).
\\
Для просторів більшої розмірності така методика кластеризації недоступна. Це пов'язано із неможливістю наглядно зобразити вхідні дані у вигляді геометричних об'єктів. 
