\chapter{Огляд стану проблеми та основні поняття}\label{ch:01}

Термін "кластерний аналіз", вперше використаний Тріоном у \cite{Tryon:Cluster:1939}, означає набір підходів та алгоритмів, призначених для об'єднання схожих об'єктів у групи. Ця технологія знайшла своє застосування в цілій низці галузей наук та є необхідною частиною більшості сучасних засобів аналізу даних.
В 1959 радянський вчений Терентьєв розробив так званий "метод кореляційних плеяд" \cite{Terentyev}, покликаний здійснювати групування на базі корелюючих ознак об'єктів. Займаючись вивченням кореляцій між різними ознаками озерної жаби, він об'єднав їх в групи за абсолютною величиною коефіцієнту кореляції. Таким чином він отримав дві групи ознак -- ознаки із великим та з малим значенням кореляції. Терентьєв назвав ці групи "кореляційними плеядами" та опублікував декілька методів їх аналізу.


Багато дослідників у своїй роботі змушені зіткнутись із необхідністю сформувати змістовні структури із набору спостережень. В біології часто постає задача розділити тварин за видами, користуючись певними вимірюваними ознаками кожної із них. Соціологи використовують техніки кластеризації для виділення соціальних груп за певними ознаками. 
