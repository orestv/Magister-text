\chapter{Огляд стану проблеми та основні поняття}\label{ch:01}

\subsection{Історія поняття}
Термін ,,кластерний аналіз'', вперше використаний Тріоном у \cite{Tryon:Cluster:1939}, означає набір підходів та алгоритмів, призначених для об'єднання схожих об'єктів у групи. Ця технологія знайшла своє застосування в цілій низці галузей наук та є необхідною частиною більшості сучасних засобів аналізу даних.
В 1959 радянський вчений Терентьєв розробив так званий ,,метод кореляційних плеяд'' \cite{Terentyev}, покликаний здійснювати групування на базі корелюючих ознак об'єктів. Займаючись вивченням кореляцій між різними ознаками озерної жаби, він об'єднав їх в групи за абсолютною величиною коефіцієнту кореляції. Таким чином він отримав дві групи ознак -- ознаки із великим та з малим значенням кореляції. Терентьєв назвав ці групи ,,кореляційними плеядами'' та опублікував декілька методів їх аналізу. Це посприяло розвитку методів кластеризації за допомогою графів. На початку 50х років також вийшли публікації Р.~Льюїса, Е.~Фікса та Дж.~Ходжеса, присвячені ієрархічним алгоритмам кластеризації. Відчутний поштовх технології кластерного аналізу дали роботи Розенблатта про розпізнаючий пристій ,,перцептрон'', котрі поклали початок теорії ,,розпізнавання без вчителя''. Поштовхом до розробки методів кластеризації стала публікація \cite{SokalSneath} в 1963 році. В своїй роботі автори виходили з того, що для створення ефективних біологічних класифікацій процедура кластеризації повинна використовувати всеможливі показники, що характеризують досліджувані організми, проводити оцінку ступеня схожості між цими організмами, та забезпечувати розташування схожих організмів в одну групу. При цьому сформовані групи повинні бути досить локальними, тобто схожість організмів всередині групи повинна бути більшою, ніж між групами. Подальший аналіз таких груп допоможе вияснити, чи відповідають вони реальним біологічним класифікаціям. Сокел та Сніт, автори цеї роботи, вважали, що виявлення структури розподілу об'єктів у групи допоможе встановити процес утворення цих груп.

\subsection{Типові задачі}
Кластеризація часто стає складовою якоїсь більшої задачі, інструментом підготовки даних для її розв'язання. Такі задачі завжди пов'язані із пошуком та виділенням змістовних структур із великих масивів даних. До них можна віднести задачі сегментування та розпізнавання зображень, мовлення, пошуку прихованих закономірностей в даних. На практиці такі задачі виникають при розв'язуванні проблем, що виникають в медицині, соціології, економіці та низці інших сфер діяльності людини. В медицині, наприклад, техніка сегментування зображення дозволяє виділяти на томограмах окремі області і на підставі їх форми та забарвлення приймати ставити діагноз. Кластеризація успішно застосовується у маркетингових дослідженнях для виявлення зв'язків між різними групами споживачів та потенційних покупців, цільових аудиторій, та оптимального позиціонування нової продукції. В соціологічних дослідженнях використовується кластеризація даних, отриманих з різних джерел, для спрощення їх подальшого аналізу. У своїй праці \cite{Zagorujko} Загоруйко описує одну із таких задач. Новосибірські вчені вивчали причини переселення людей з сіл в міста. Були вислані експедиції в навколишні села, жителям яких задавали приблизно сто анкетних питань, що стосувались віку, сімейного становища, освіти та ін. Після завершення опитування дослідники постали перед необхідністю аналізувати більш ніж сім тисяч анкет, що містили понад сто питань кожна. Ці дані було введено в програму таксономії, котра повернула сім великих таксонів, середні характеристики яких дозволили дати зібраним даним змістовну інтерпретацію. Наприклад, виділився кластер, що містив переважно жінок середнього віку, котрі мали дорослих дітей в місті. Очевидно, представниці цього таксону, названого дослідниками "бабусі", їхали в місто доглядати за своїми внуками. Решта таксонів опрацьовано аналогічно.
