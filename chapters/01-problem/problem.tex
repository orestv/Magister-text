\chapter{Огляд стану проблеми та основні поняття}\label{ch:01}

Постановку задачі кластеризації можна знайти ще в "Листі вченому сусіду" Демокріта. Тут він пише, "Якщо тобі, дорогий друже, потрібно розібратись у складному нагромадженні фактів чи речей, ти спершу розклади їх на декілька купок за схожістю. Ситуація проясниться і ти зрозумієш природу цих речей". 
Сьогодні задачі кластеризації супроводжують більшість задач аналізу даних. початок такої тенденції співпадає із початком розвитку обчислювальної техніки -- останній дав змогу застосовувати техніки кластеризації на таких масивах даних, які не піддавались обробці вручну.
